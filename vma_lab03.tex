\documentclass[12pt]{report}

\usepackage[russian]{babel}
\usepackage[utf8]{inputenc}
\usepackage{amsmath}
\usepackage{amssymb}
\usepackage{tikz}
\usepackage{caption}
\usepackage[a4paper,margin=1.0in,footskip=0.25in]{geometry}
\usepackage{listings}  
\usepackage{color}
\usepackage{tabto}

\definecolor{dkgreen}{rgb}{0,0.6,0}
\definecolor{gray}{rgb}{0.5,0.5,0.5}
\definecolor{mauve}{rgb}{0.58,0,0.82}

\lstset{frame=tb,
  language=Python,
  aboveskip=3mm,
  belowskip=3mm,
  showstringspaces=false,
  columns=flexible,
  basicstyle={\small\ttfamily},
  numbers=none,
  numberstyle=\tiny\color{gray},
  keywordstyle=\color{blue},
  commentstyle=\color{dkgreen},
  stringstyle=\color{mauve},
  breaklines=true,
  breakatwhitespace=true,
  tabsize=3,
  frame=none
}

%Title
\title{\vspace{-3cm}Лабораторная №3}
\author{Жидович Максим, группа №1}
\date{8 ноября 2021}

\renewcommand\thesection{\arabic{section}}

\begin{document}

\maketitle

\section{Постановка задачи}

Разработать программу численного решения СЛАУ методом левой прогонки
\[
\begin{bmatrix}
a_{11} & a_{12}  & \cdots   & a_{1m}   \\
a_{21} & a_{22}  & \cdots   & a_{2m}  \\
\vdots & \vdots  & \ddots   & \vdots  \\
a_{n1} & a_{n2}  & \cdots\  & a_{nm}  \\
\end{bmatrix}
\begin{bmatrix}
x_{1} \\
x_{2} \\
\vdots \\
x_{n} \\
\end{bmatrix}
=
\begin{bmatrix}
b_{1} \\
b_{2} \\
\vdots \\
b_{n} \\
\end{bmatrix}
\]

\section{Входные данные}

В работе использовались:

Трёхдиагональная матрица $A$ размерности 10 сформированная следующим образом:
\[
a_{ii} = 5(i+1)^{p/2}, i = \overline{0,10},
\]
\[
a_{ij} = (i + 1)^{p/2} + j^{q/2},  i \neq j.
\]
при $p = 1$ и $q = 1$.
\[A = 
\begin{bmatrix}
    5 & 2 & 0 & 0 & 0 & 0 & 0 & 0 & 0 & 0 \\
    1.4142 & 7.071 & 2.8284 & 0 & 0 & 0 & 0 & 0 & 0 & 0 \\
    0 & 2.732 & 8.6602 & 3.4641 & 0 & 0 & 0 & 0 & 0 & 0 \\
    0 & 0 & 3.4142 & 10 & 4 & 0 & 0 & 0 & 0 & 0 \\
    0 & 0 & 0 & 3.9681 & 11.1803 & 4.4721 & 0 & 0 & 0 & 0 \\
    0 & 0 & 0 & 0 & 4.4495 & 12.2474 & 4.899 & 0 & 0 & 0 \\
    0 & 0 & 0 & 0 & 0 & 4.8818 & 13.2288 & 5.2915 & 0 & 0 \\
    0 & 0 & 0 & 0 & 0 & 0 & 5.278 & 14.1421 & 5.6569 & 0 \\
    0 & 0 & 0 & 0 & 0 & 0 & 0 & 5.6458 & 15 & 6 \\
    0 & 0 & 0 & 0 & 0 & 0 & 0 & 0 & 5.9907 & 15.8114 \\
\end{bmatrix}
\]

Вектор $x = (m, m+1, \ldots, n+m-1)$, полученный при $m = 4, n = 10$:
\[
\begin{bmatrix}
   4 & 5 & 6 & 7 & 8 & 9 & 10 & 11 & 12 & 13  
\end{bmatrix}
\]

Вектор $b$, полученный умножением $A$ на $x$:
\[
\begin{bmatrix}
   25 & 57.9828 & 89.8705 & 122.4853 & 157.4688 & 194.8128 & 234.4305 & 276.2249 & 320.1033 & 277.4365
\end{bmatrix}
\]


\section{Краткая теория}


Метод Гаусса является универсальным прямым методом решения систем линейных алгебраических уравнений. В то же время, как мы уже видели на примере решения СЛАУ с симметричной матрицей, учет специфики задачи позволяет построить алгоритмы с меньшими, по сравнению с универсальными, вычислительными затратами. Рассмотрим метод прогонки – один из важнейших для приложений методов решения СЛАУ. 

Метод прогонки есть метод исключения Гаусса без выбора главного элемента, примененный к системе уравнений с трехдиагональной матрицей.

Алгоритм метода левой прогонки для решения системы имеет следующий вид:
\begin{enumerate}
	\item Прямая прогонка – вычисление прогоночных коэффициентов по формулам (7) файла «Метод прогонки»
	\item Обратная прогонка – вычисление решения по формулам (8) файла «Метод прогонки»
\end{enumerate}

\section{Листинг программы}

\lstset{language=Python}
\lstset{extendedchars=\true}

Код программы, реализующий $LDL^{T}$-разложение: 

\begin{lstlisting}
def progonka(A, b, need_matrix=False):
	"""Solve system of linear equations using tridiagonal matrix algorithm"""

	n = len(A)
	y = [0] * n
	ksi = [0] * n
	eta = [0] * n

	# find coefficients
	ksi[n - 1] = -1 * A[n - 1][n - 2] / A[n - 1][n - 1]
	eta[n - 1] = b[n - 1] / A[n - 1][n - 1]

	for i in range(n - 2, 0, -1):
		tmp = A[i][i] + A[i][i + 1] * ksi[i + 1]
		ksi[i] = -1 * A[i][i - 1] / tmp
		eta[i] = (b[i] - A[i][i + 1] * eta[i + 1]) / tmp
	eta[0] = (b[0] - A[0][1] * eta[1]) / (A[0][0] + A[0][1] * ksi[1])

	# find solution
	y[0] = eta[0]
	for i in range(n - 1):
		y[i + 1] = ksi[i + 1] * y[i] + eta[i + 1]


	# get matrix after direct run
	if need_matrix is True:
		matrix = [[0] * n for i in range(n)]
		matrix[0][0] = 1
		for i in range(1, n):
			matrix[i][i - 1] = -1 * ksi[i]
			matrix[i][i] = 1
		v = [eta[i] for i in range(n)]

		return matrix, v, y

	return y

\end{lstlisting}

\section{Выходные данные}

Преобразованная матрица $A$ после прямой прогонки: 
\[
\begin{bmatrix}
     1 & 0 & 0 & 0 & 0 & 0 & 0 & 0 & 0 & 0 \\
     0.2356 & 1 & 0 & 0 & 0 & 0 & 0 & 0 & 0 & 0 \\
     0 & 0.3778 & 1 & 0 & 0 & 0 & 0 & 0 & 0 & 0 \\
     0 & 0 & 0.4126 & 1 & 0 & 0 & 0 & 0 & 0 & 0 \\
     0 & 0 & 0 & 0.4314 & 1 & 0 & 0 & 0 & 0 & 0 \\
     0 & 0 & 0 & 0 & 0.4432 & 1 & 0 & 0 & 0 & 0 \\
     0 & 0 & 0 & 0 & 0 & 0.4509 & 1 & 0 & 0 & 0 \\
     0 & 0 & 0 & 0 & 0 & 0 & 0.4537 & 1 & 0 & 0 \\
     0 & 0 & 0 & 0 & 0 & 0 & 0 & 0.4436 & 1 & 0 \\
     0 & 0 & 0 & 0 & 0 & 0 & 0 & 0 & 0.3789 & 1 \\
\end{bmatrix}
\]

Вектор приближённого решения:
\[x^* = 
\begin{bmatrix}
     3.(9) & 5 & 6 & 6.(9) & 8 & 8.(9) & 10 & 10.(9) & 12 & 12.(9)
\end{bmatrix}
\]

Относительная погрешность: $4.099285014000578 \cdot 10^{-16}$

\section{Выводы}

\tabМетод прогонки является прямым методом, его точность  сопоставима с точностью метода Гаусса. Погрешности близки к нулю (порядок $10^{-16}$), это говорит о том, что ответ достаточно точен.

\end{document}
