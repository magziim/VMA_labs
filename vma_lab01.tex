\documentclass[12pt]{report}

\usepackage[russian]{babel}
\usepackage[utf8]{inputenc}
\usepackage{amsmath}
\usepackage{amssymb}
\usepackage{tikz}
\usepackage{caption}
\usepackage[a4paper,margin=1.0in,footskip=0.25in]{geometry}
\usepackage{listings}  
\usepackage{color}
\usepackage{tabto}

\definecolor{dkgreen}{rgb}{0,0.6,0}
\definecolor{gray}{rgb}{0.5,0.5,0.5}
\definecolor{mauve}{rgb}{0.58,0,0.82}

\lstset{frame=tb,
  language=Python,
  aboveskip=3mm,
  belowskip=3mm,
  showstringspaces=false,
  columns=flexible,
  basicstyle={\small\ttfamily},
  numbers=none,
  numberstyle=\tiny\color{gray},
  keywordstyle=\color{blue},
  commentstyle=\color{dkgreen},
  stringstyle=\color{mauve},
  breaklines=true,
  breakatwhitespace=true,
  tabsize=3,
  frame=none
}

%Title
\title{\vspace{-3cm}Лабораторная №1\\ \vspace{0.5cm}Вариант 1}
\author{Жидович Максим, группа №1}
\date{28 сентября 2021}

\renewcommand\thesection{\arabic{section}}

\begin{document}

\maketitle

\section{Постановка задачи}

Необходимо реализовать метод Гаусса для произвольной СЛАУ $Ax = b$ без выбора ведущего элемента, с выбором ведущего элемента по строке.
\[
\begin{bmatrix}
a_{11} & a_{12}  & \cdots   & a_{1m}   \\
a_{21} & a_{22}  & \cdots   & a_{2m}  \\
\vdots & \vdots  & \ddots   & \vdots  \\
a_{n1} & a_{n2}  & \cdots\  & a_{nm}  \\
\end{bmatrix}
\begin{bmatrix}
x_{1} \\
x_{2} \\
\vdots \\
x_{n} \\
\end{bmatrix}
=
\begin{bmatrix}
b_{1} \\
b_{2} \\
\vdots \\
b_{n} \\
\end{bmatrix}
\]
Предполагается, что $detA \ne 0$. Тогда решение системы существует и оно единственно.

Необходимо найти:
\begin{enumerate}
	\item Приближённые решения $x_{1}^*, x_{2}^*$
	\item Относительные погрешности $\frac{\|x_{1}^* - x\|_{\infty}}{\|x\|_{\infty}}, \frac{\|x_{2}^* - x\|_{\infty}}{\|x\|_{\infty}}$
\end{enumerate}

\section{Краткая теория}

Алгоритм решения СЛАУ методом Гаусса подразделяется на два этапа.
\begin{enumerate}
    \item На первом этапе осуществляется так называемый прямой ход, когда путём элементарных преобразований над строками систему приводят к ступенчатой или треугольной форме. Для этого ненулевые элементы первого столбца всех нижележащих строк обнуляются путём вычитания из каждой строки первой строки, домноженной на отношение первого элемента этих строк к первому элементу первой строки. После того, как указанные преобразования были совершены, первую строку и первый столбец мысленно вычёркивают и продолжают, пока не останется матрица нулевого размера.
    \item На втором этапе осуществляется так называемый обратный ход, суть которого заключается в том, чтобы выразить все получившиеся базисные переменные через небазисные и построить фундаментальную систему решений, либо, если все переменные являются базисными, то выразить в численном виде единственное решение системы линейных уравнений. Эта процедура начинается с последнего уравнения, из которого выражают соответствующую базисную переменную (а она там всего одна) и подставляют в предыдущие уравнения, и так далее, поднимаясь по «ступенькам» наверх. Каждой строчке соответствует ровно одна базисная переменная, поэтому на каждом шаге, кроме последнего (самого верхнего), ситуация в точности повторяет случай последней строки.
\end{enumerate}
\tabАлгоритм решения СЛАУ методом Гаусса с выбором ведущего элемента по строке отличается от вышеприведённого. На $i$-ом шаге прямого хода необходимо определить $\max\limits_{1 \le j \le m}|a_{ij}|$ и путём перестановки поменять между собой 1 и $j$ столбцы. При этом важной частью алгоритма является сохранение перестановок с целью приведения вектора решений в изначальный вид.

\section{Листинг программы}

\lstset{language=Python}
\lstset{extendedchars=\true}

Код программы, реализующей алгоритм метода Гаусса: 

\begin{lstlisting}
def gauss_method(a, b):
    """Function return solution(x) of a system of equations ax = b"""
    n = len(a)   # size of matrix

    # forward stroke 
    for k in range(n):
        for i in range(k + 1, n):
            factor = a[i][k] / a[k][k]
            for j in range(k, n):
                a[i][j] -= a[k][j] * factor
            b[i] -= b[k] * factor

    x = [0] * n   # solution
    
    # reverse stroke
    for k in range(n - 1, -1, -1):
        sum_of_previous_x = 0
        for i in range(k + 1, n):
            sum_of_previous_x += a[k][i] * x[i]
        x[k] = (1 / a[k][k]) * (b[k] - sum_of_previous_x)

    return x
\end{lstlisting}

Код программы, реализующей алгоритм метода Гаусса с выбором ведущего элемента по строке: 

\begin{lstlisting}
def gauss_method_with_main_element(a, b):
    """Function return solution(x) of a system of equations ax = b"""
    n = len(a)   # size of matrix
    
    changed_a = a.copy()
    swaps_x = [i for i in range(n)]   # vector of `x` swaps

    # forward stroke
    for k in range(n):
        # find index of max element 
        max_index = changed_a[k].index(max(changed_a[k], key=abs))
        
        # swap `max element` column with current
        for c in range(n):
            changed_a[c][max_index], changed_a[c][k] = changed_a[c][k], changed_a[c][max_index]
        swaps_x[k], swaps_x[max_index] = swaps_x[max_index], swaps_x[k]

        for i in range(k + 1, n):
            factor = changed_a[i][k] / changed_a[k][k]
            for j in range(k, n):
                changed_a[i][j] -= changed_a[k][j] * factor
            b[i] -= b[k] * factor

    x = [0] * n   # solution
    
    # reverse stroke
    for k in range(n - 1, -1, -1):
        sum_of_previous_x = 0
        for i in range(k + 1, n):
            sum_of_previous_x += changed_a[k][i] * x[swaps_x[i]]
        x[swaps_x[k]] = (1 / changed_a[k][k]) * (b[k] - sum_of_previous_x)

    return x
\end{lstlisting}

\section{Выходные данные}

\tabМатрица А размерности 10х10 сгенерированная случайным образом:
\[
\begin{bmatrix}
      -121 &      -247 &       575 &      -817 &      -383 &       437 &       530 &      -459 &       870 &      -158 \\
      -210 &       712 &       155 &      -179 &       752 &      -229 &      -109 &       285 &       805 &      -563 \\
       662 &       857 &      -532 &       175 &       559 &      -460 &      -907 &       602 &       984 &       267 \\
      -436 &      -778 &       759 &       990 &      -114 &      -371 &      -938 &       -42 &       -78 &       145 \\
      -858 &      -475 &      -131 &      -142 &      -630 &       785 &       453 &      -925 &       144 &       890 \\
       412 &      -790 &      -452 &       492 &      -152 &      -214 &       519 &       780 &      -836 &      -291 \\
      -471 &       653 &       318 &       -23 &      -709 &      -331 &      -434 &       798 &      -796 &      -303 \\
      -267 &      -956 &      -732 &      -160 &      -831 &      -361 &      -871 &       339 &       -29 &      -600 \\
       276 &      -947 &        -5 &      -714 &      -924 &       668 &       152 &       273 &      -681 &       659 \\
       316 &       579 &      -607 &      -529 &      -324 &        67 &      -693 &       142 &       906 &      -265 \\
\end{bmatrix}
\]

Вектор $Х$ размерности 10 сгенерированный случайным образом:
\[
\begin{bmatrix}
   -841 & -102 & 758 & -948 & 86 & -448 & 790 & 515 & 873 & 925   
\end{bmatrix}
\]

Полученный вектор $f$ при решении системы $f = Ax$:
\[
\hspace{-1cm}
\begin{bmatrix}
   1904282 & 801087 & -259651 & -457381 & 1229943 & -1179437 & -227406 & -1084677 & 434569 & -269835
\end{bmatrix}
\]

Вектор $x_{1}^*$ размерности 10 полученный при использовании обычного метода Гаусса:
\[
\begin{bmatrix}
   -841 & -101.999 & 758 & -948 & 85.999 & -448 & 790 & 515 & 872.999 & 925
\end{bmatrix}
\]

Вектор $x_{2}^*$ размерности 10 полученный при использовании метода Гаусса с выбором ведущего элемента по строке:
\[
\begin{bmatrix}
   -841 & -102 & 758 & -948 & 86 & -448 & 790 & 515 & 872.999 & 925
\end{bmatrix}
\]

Относительная огрешность полученная при использовании обычного метода Гаусса: $8.694404783562407e-15$

Относительная погрешность полученная при использовании метода Гаусса с выбором ведущего элемента: $3.1179934396223807e-15$

\section{Выводы}

\tabПогрешности, полученные при использовании обычного метода Гаусса и метода Гаусса с выбором ведущего элемента, близки к нулю (порядок $10^{-15}$), это говорит о том, что ответ достаточно точен.

\end{document}
